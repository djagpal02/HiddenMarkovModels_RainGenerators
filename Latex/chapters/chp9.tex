Our testing in \ref{Results} suggests the HMM on its own does have potential. The model matched both the training and test data well. While it did struggle to explain extreme rainfall, this could be an area for further research. Furthermore, for our testing, we used only three random starts of Baum-Welch. Considering the incredibly high complexity of the parameter space, due to large matrices, one can expect there to be many more local optima than just three. As such, increasing this number would likely increase the likelihood of achieving a better fit by finding a superior optimum.  Thus we are inclined to believe the performance of HMM rainfall models could be further improved.

Our generalised model was not successful. The model itself was chosen based on purely natural phenomenon and thus does not dismiss the idea that a generalised HMM model may be helpful. If we can find a function of random variables that perfectly fit the rows of the observation matrix, then hypothetically, we can expect matching performance to that of the HMM only model.  While the HMM fits the observed data more closely, a generalised model will likely allow for improvements in out-of-sample testing, particularly at the extremes. 

We tested a three-state model. However, this was an arbitrary choice. Future research may consider increasing this number as it may allow for a superior fit to the data. This change, of course, must be done after careful considerations of computational cost as the algorithms used in fitting are extremely sensitive to the increase in value of this parameter.

To conclude, can hidden Markov models be used to simulate rainfall? From our testing, we have insufficient evidence to suggest not. Further research into the three areas listed above combined with testing performance against other models should determine whether HMMs provide any advantage.