In this section, we will briefly visit the foundations on which we will build throughout this paper. For most, this will be a simple refresher.

\section{Mathematical Foundations}
We start with a few key mathematical concepts.
\subsection{Probability Theory}
To discuss any probabilistic ideas, we must first understand general probability theory. We will start by defining a probability space.
\begin{definition}
    \label{probtheory}
    Probability Space \\
    A probability space is defined by $(\Omega ,\sigalgebra , \prob)$. $\Omega$ is the non-empty set of all possible outcomes, such that all events $\omega$ $\in$ $\Omega$. $\prob$ is a probability measure, a function $\prob$(A) that maps event A to a number within [0,1] based on the likelihood of the event. $\sigalgebra$ is a $\sigma$-algebra on $\Omega$ if

    \begin{enumerate}
        \item $\Omega$ $\in$ $\sigalgebra$
        \item A $\in$ $\sigalgebra$ implies $A^c$ $\in$ $\sigalgebra$
        \item if $A_1, A_2, A_3$,... are in $\sigalgebra$ then so is $A_1 \cup A_2 \cup A_3$...
    \end{enumerate}
\end{definition}

\subsection{Conditional Probability}
Sometimes we require the probability of an event assuming another event has occurred. In such situations, we require a conditional probability. Given two events A and B, the probability of event A occurring conditioned on event B: \\ 
\begin{equation}
\label{condprob}
    \prob (A|B) = \dfrac{\prob (A \cap B)}{\prob (B)}, \  \forall A \in \sigalgebra
\end{equation}
\\

From \ref{condprob} and the fact that for dependednt events $\prob$(A $\cap$ B) = $\prob$(B $\cap$ A) we can see that:

\begin{equation}
\label{intersection}
    \prob (A \cap B) = \prob (A|B) \prob (B) = \prob (B|A) \prob (A), \  \forall A,B \in \sigalgebra
\end{equation}
\\

Subsituting \ref{intersection} into \ref{condprob} we get the famous Bayes Theorem. 

\begin{theorem}
\label{bayes}
    Bayes’ Theorem \\
    For dependent events A and B with probability space $(\Omega ,\sigalgebra , \prob)$ , where $\prob$         (B) $\ne$ 0,
    \begin{equation}
        \prob (A|B) = \dfrac{\prob (B|A) \prob (A)}{\prob (B)}, \  \forall A \in \sigalgebra
    \end{equation}
\end{theorem}



\subsection{Stochastic Process}
To be able to define a Markov model, of any kind, we must first define a stochastic process. 
\begin{definition}
\label{stochasticp} 
    Stochastic Process \\
    Given an ordered set T and probability space $(\Omega ,\sigalgebra , \prob)$ a stochastic process is        a collection of random variables X = \{$X_t$; t$\in$T\}. Based on t $\in $ T and $\omega \in \Omega$        we get a numerical realization of the process.  For simplicity, this may be viewed as a function;           $X_t( \omega)$.

\end{definition}




\section{Applied Foundations}