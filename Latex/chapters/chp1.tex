Most natural systems are black-box systems. We can observe their output but have little information regarding the inputs that decide the output,  making modelling such processes extremely difficult.  Hidden Markov Models (HMMs) offer a method to help us to explain some black-box systems given the assumption that the underlying system has Markovian states.

Rainfall is an example of one such black-box system. We can observe the amount of rainfall but have limited information regarding why we observed a given amount of rain.  Accurate simulations, among other things,  can allow for an improved understanding of areas with a higher risk for flood or drought. This information can allow residents of the given areas to prepare for such eventualities.

Extensive research into this area has been conducted by \cite{Cowpertwait1994}. He models rainfall as collections of 2-dimensional discs. There is a large outer disc that contains smaller discs with correlated rainfall intensities. \cite{Grando2019} extends on Cowpertwaits ideas by utilising HMMs state system for the correlations between smaller discs. She simplifies Cowpertwait's model and explores parameter fitting methods. We aim to expand on Grando's ideas, fit HMM for rainfall and decide whether HMM provide any use in modelling rainfall.

In Chapter,\ref{Hidden_Markov} we begin by defining HMMs and breaking down three key problems; evaluation \ref{Hidden_Markov:Evaluation}, decoding \ref{Hidden_Markov:Decoding} and learning \ref{Hidden_Markov:Learning}. Answering these questions helps build a deeper understanding of applications of HMM and provides us with a set of algorithms. In Chapter \ref{Model_Selection}, we continue to develop key ideas beginning with maximum likelihood estimators \ref{Model_Selection:Maximum_Liklihood_Estimators}; a more general approach to selecting the ideal model.  We also introduce Absolute Bayesian Computation \ref{Model_Selection:Approximate_Bayesian_Computation}, one of Grando's model-fitting methods in \cite{Grando2019}. Finally, we introduce the Kolmogorov-Smirnov test \ref{Model_Selection:Kolmogorov_Smirnov_Test}; a hypothesis Test to help decide whether two samples come from the same continuous distributions.

We begin our research by replicating Grando's results in Chapter \ref{Replicating_Existing_Rainfall_Model}. This provides us with a baseline to branch our research. We implement her ideas in C++, making use of multithreading, to run her simulations in 3-5 minutes rather than 15-40 hours. This gives us the opportunity to run multiple simulations and detect errors she had missed due to being unable to do so. A separate analysis of the results can be found among the code files. 

We then breakdown issues faced by Grando's Model in Chapter \ref{Extending_HMM_Based_Rainfall_Models}. These are then addressed in the proposed method of implementing HMMs to rainfall in Chapter \ref{Simple_Rainfall_HMM}. We propose a further modification to our implementation, simplifying the model further, through simplifying the HMMs observation matrix,  in Chapter \ref{Generalising_the_Observation_Matrix}. We test our two implementations by comparing simulations with the recorded data in Chapter \ref{Results}. We Finally conclude our research and propose methods of extending in Chapter \ref{Future_Research}.


All programs have been run on the personal desktop (AMD Ryzen™ 9 3900X with 12 Cores and 24 Threads running from 3.8GHz up to 4.6GHz) using C++20. Analysis has been conducted using Python 3 on Jupyter Notebooks as well as R. The data used is German Rainfall data, used by Grando \cite{Grando2019}, to ensure consistency in comparison. All programs, code and analysis, can be found on the \href{https://github.com/djagpal02/HiddenMarkovModels_RainGenerators }{Github} page. The \href{https://github.com/djagpal02/HiddenMarkovModels_RainGenerators }{Github} contains additional analysis not presented in this dissertation to avoid repetition. 

