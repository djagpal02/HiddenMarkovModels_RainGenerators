In this chapter we will extend on Grando's ideas and adapt her model.


\section{Disk Structure}
Cowpertwait's disk structure for rain models is derived from nature. This makes it extremely appealing as a model. Giving Grando good reason to translate these ideas to her HMM based model. However, for our goal of rain simulation, her model has some unnessary complexity. 

Grando calculates the number of storm discs over the entire space (all sites) but then only selects discs above the current site. She repeats this for all sites. We can expect most storm discs will not cover all sites, suggesting this method will generate a large number of potential discs and then reject most of them. This is extremly ineffcient and computationaly expensive. To solve this issue one may propose the following:
\begin{itemize}
    \item Simulate the number and size of discs over the entire map of sites and deterimine which storms are above which sites. This method stops the unnessary iteration through all sites and provides the same results as the paramter estimates will be account for this change.
    \item Simulate the number and size of discs over all sites independently. Here we ignore the spatial data of the sites. This prevents us from understanding which sites are covered by the same discs, but this is not of importance to us and thus can be left as is. Furthermore, we can remove the assumption of the shape of the disc as it is no longer relevant, allowing our model to be further generalised. This allows us to remove the radius parameters for each state, reducing total parameters by the number of states.
\end{itemize}

While both methods mentioned above allow for improvement, ultimately it does not matter which we pick. The model will fit the given parameters and the orientation of the parameters does not make a difference as they will adjust to compensate, resulting in a model that produces a similar output. As such, for our model we decide to go with the second method. Since we will be calculating over each site independently, we decide to calculate parameters for each seperately rather than combining these results. 



\section{Split monthly}

Rainfall is not consistent throughout the year. A simulation of rainfall in summer would likely have signficiatnly different statistical properties to one from the winter. Grando's Model does not account for this, as such a simulation using this model could produce an output more likely in winter or summer, one cannot tell. 

To address these concerns, we utilise Cowpertwaits **ref** methodology, spliting the data into months. This allows for simulating for particular months as well as a better fit for the overall data. Since our data is a timeseries of rainfall amounts with not much other information given we divide our data into the average number of days in a month; 30. We then label the first twelve groups 0 to 11 and fill each group cyclying through 0-11. The exact methodology and jupyter notebook can be found in the "DataPrep" folder under "MyCode" within the code files. 

Another nessesary modeification was changing the missing values. The given data contained "-9999" for any days with missing data. Removing these values could create unequal sized datasets for each month. To avoid this we decided to set these values to a particular number. Unfortuently any value we select will cause the data mean to shift towards the value. To solve this we set this value as the mean of the months rainfall ignoring the missing data. 

\section{High dimensionality}

HMMs with even a few states contain a large number of paramters. Grando's Model contains 21. Through our adjustments made to the disc structure, we have reduced this to 18. High dimensionality is a common problem but Unfortuently does not have straightforward solutions. We could introduce new assumptions and try to simplify the data using dimensionality reduction methods but this may limit our ability to fit our model well. 

To address this problem, we revert back to chapter 4 \ref{chp4}. Instead of approaching this problem by starting the parameter estimation from the non-hmm parameters, we approach from HMM parameters. Momentarily ignoring the non-HMM paramters, we know that Baum-Welch can be used to estimate 12 of these 18 parameters. While this is not all, this is a great decrease in dimensions. However, our model does not have an observation matrix, which is required by the Baum-Welch **ref** algorithm. We address this in the following chapter.

